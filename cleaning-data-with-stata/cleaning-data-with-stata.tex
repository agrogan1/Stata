\documentclass[]{article}
\usepackage{lmodern}
\usepackage{stata}
\usepackage{amssymb,amsmath}
\usepackage{ifxetex,ifluatex}
\usepackage{fixltx2e} % provides \textsubscript
\ifnum 0\ifxetex 1\fi\ifluatex 1\fi=0 % if pdftex
  \usepackage[T1]{fontenc}
  \usepackage[utf8]{inputenc}
\else % if luatex or xelatex
  \ifxetex
    \usepackage{mathspec}
    \usepackage{xltxtra,xunicode}
  \else
    \usepackage{fontspec}
  \fi
  \defaultfontfeatures{Mapping=tex-text,Scale=MatchLowercase}
  \newcommand{\euro}{€}
\fi
% use upquote if available, for straight quotes in verbatim environments
\IfFileExists{upquote.sty}{\usepackage{upquote}}{}
% use microtype if available
\IfFileExists{microtype.sty}{%
\usepackage{microtype}
\UseMicrotypeSet[protrusion]{basicmath} % disable protrusion for tt fonts
}{}
\usepackage[margin=1 in]{geometry}
\ifxetex
  \usepackage[setpagesize=false, % page size defined by xetex
              unicode=false, % unicode breaks when used with xetex
              xetex]{hyperref}
\else
  \usepackage[unicode=true]{hyperref}
\fi
\usepackage[usenames,dvipsnames]{color}
\hypersetup{breaklinks=true,
            bookmarks=true,
            pdfauthor={Andy Grogan-Kaylor},
            pdftitle={Cleaning Data With Stata},
            colorlinks=true,
            citecolor=blue,
            urlcolor=blue,
            linkcolor=magenta,
            pdfborder={0 0 0}}
\urlstyle{same}  % don't use monospace font for urls
\setlength{\parindent}{0pt}
\setlength{\parskip}{6pt plus 2pt minus 1pt}
\setlength{\emergencystretch}{3em}  % prevent overfull lines
\providecommand{\tightlist}{%
  \setlength{\itemsep}{0pt}\setlength{\parskip}{0pt}}
\setcounter{secnumdepth}{0}

\title{Cleaning Data With Stata}
\author{Andy Grogan-Kaylor}
\date{18 Feb 2021 15:47:48}

% Redefines (sub)paragraphs to behave more like sections
\ifx\paragraph\undefined\else
\let\oldparagraph\paragraph
\renewcommand{\paragraph}[1]{\oldparagraph{#1}\mbox{}}
\fi
\ifx\subparagraph\undefined\else
\let\oldsubparagraph\subparagraph
\renewcommand{\subparagraph}[1]{\oldsubparagraph{#1}\mbox{}}
\fi

\begin{document}
\maketitle

\section{Background}\label{background}

It sometimes seems like 80\% of our work as data analysts is cleaning
the data, while only 20\% is the actual analysis. Here are some Stata
commands that are useful in cleaning data.

\begin{quote}
First we simulate some data to work with, and to clean.
\end{quote}

\section{Simulate Some Data}\label{simulate-some-data}

\begin{stlog}
. clear all 
\end{stlog}

\begin{stlog}
. set obs 100 // 100 observations
number of observations (_N) was 0, now 100
\end{stlog}

\begin{stlog}
. generate id = _n // random id 
\end{stlog}

\begin{stlog}
. generate age = rnormal(50,10) // random generated age
\end{stlog}

\begin{stlog}
. replace age = 200 in 1 // someone is 200 years old!
(1 real change made)
\end{stlog}

\begin{stlog}
. generate happy = runiformint(1,5) // randomly generated happiness
\end{stlog}

\begin{stlog}
. replace happy = 999 in 10 // simulate a missing value
(1 real change made)
\end{stlog}

\begin{stlog}
. generate somethingelse = rnormal(0, 1) // something else!
\end{stlog}

\section{Look At Some Of The Data}\label{look-at-some-of-the-data}

\begin{stlog}
. list in 1/10 // list first 10 observations
{\smallskip}
     {\TLC}\HLI{35}{\TRC}
     {\VBAR} id        age   happy   somethi{\tytilde}e {\VBAR}
     {\LFTT}\HLI{35}{\RGTT}
  1. {\VBAR}  1        200       1    1.497365 {\VBAR}
  2. {\VBAR}  2   66.57332       5    1.376741 {\VBAR}
  3. {\VBAR}  3   34.82442       3   -1.079351 {\VBAR}
  4. {\VBAR}  4   50.09281       2   -.0128491 {\VBAR}
  5. {\VBAR}  5   33.88482       5   -.0264842 {\VBAR}
     {\LFTT}\HLI{35}{\RGTT}
  6. {\VBAR}  6   61.31433       3    .4895257 {\VBAR}
  7. {\VBAR}  7   53.06971       1   -.1926064 {\VBAR}
  8. {\VBAR}  8   54.21424       5   -2.171089 {\VBAR}
  9. {\VBAR}  9   53.80603       3    -.314195 {\VBAR}
 10. {\VBAR} 10   48.48671     999   -.1993797 {\VBAR}
     {\BLC}\HLI{35}{\BRC}
\end{stlog}

\section{Clean The Data!}\label{clean-the-data}

\subsection{Look at Variables}\label{look-at-variables}

When we look at variables we are looking for values that don't make
sense, or that are outside the plausible range.

\begin{stlog}
. describe // describe the data
{\smallskip}
Contains data
  obs:           100                          
 vars:             4                          
\HLI{150}
              storage   display    value
variable name   type    format     label      variable label
\HLI{150}
id              float   \%9.0g                 
age             float   \%9.0g                 
happy           float   \%9.0g                 
somethingelse   float   \%9.0g                 
\HLI{150}
Sorted by: 
     Note: Dataset has changed since last saved.
\end{stlog}

\begin{stlog}
. summarize // descriptive statistics
{\smallskip}
    Variable {\VBAR}        Obs        Mean    Std. Dev.       Min        Max
\HLI{13}{\PLUS}\HLI{57}
          id {\VBAR}        100        50.5    29.01149          1        100
         age {\VBAR}        100    49.90129    17.77663    24.7626        200
       happy {\VBAR}        100       12.96    99.61034          1        999
somethinge{\tytilde}e {\VBAR}        100    .0205791    1.054944  -2.456728   3.322945
\end{stlog}

\begin{stlog}
. tabulate happy // tabulation of this particular categorical variable
{\smallskip}
      happy {\VBAR}      Freq.     Percent        Cum.
\HLI{12}{\PLUS}\HLI{35}
          1 {\VBAR}         20       20.00       20.00
          2 {\VBAR}         20       20.00       40.00
          3 {\VBAR}         21       21.00       61.00
          4 {\VBAR}         16       16.00       77.00
          5 {\VBAR}         22       22.00       99.00
        999 {\VBAR}          1        1.00      100.00
\HLI{12}{\PLUS}\HLI{35}
      Total {\VBAR}        100      100.00
\end{stlog}

Notice that\ldots{}

\begin{itemize}
\tightlist
\item
  There are variables in which we may not have interest.
\item
  None of the variables are labelled informatively.
\item
  Variables do not seem to have informative value labels.
\item
  Someone appears to 200 years old.
\item
  There appear to be missing values in the variable \texttt{happy} that
  need to be recoded.
\end{itemize}

\subsection{\texorpdfstring{Only \texttt{keep} The Variables Of
Interest}{Only keep The Variables Of Interest}}\label{only-keep-the-variables-of-interest}

We may only be interested in keeping some variables to keep our analytic
data set more manageable.

For this particular analysis we may wish to drop the variable called
\texttt{somethingelse}.

\begin{stlog}
. keep id age happy // keep only relevant variables
\end{stlog}

We could also have said \texttt{drop\ somethingelse}.

\subsection{\texorpdfstring{Add \emph{Variable}
Labels}{Add Variable Labels}}\label{add-variable-labels}

\begin{stlog}
. label variable id "ID" // label variable
\end{stlog}

\begin{stlog}
. label variable age "Age in Years" // label variable
\end{stlog}

\begin{stlog}
. label variable happy "Happiness Scale" // label variable
\end{stlog}

\subsection{\texorpdfstring{Create \emph{Value}
Labels}{Create Value Labels}}\label{create-value-labels}

\begin{stlog}
. label define happy 1 "Rarely" 2 "Sometimes" 3 "Often" 4 "Always" // create value label
\end{stlog}

\subsection{\texorpdfstring{Attach \emph{Value} Labels To
\emph{Variables}}{Attach Value Labels To Variables}}\label{attach-value-labels-to-variables}

\emph{Variables} and \emph{value labels} can have the same names but are
different things. We add the variable label \texttt{happy} to the
variable named \texttt{happy}.

\begin{stlog}
. label values happy happy // attach VALUE LABEL happy to VARIABLE happy
\end{stlog}

\subsection{\texorpdfstring{Recode Outliers or Values That Are Errors
(\texttt{recode})}{Recode Outliers or Values That Are Errors (recode)}}\label{recode-outliers-or-values-that-are-errors-recode}

\begin{stlog}
. recode happy (999 = .) // recode values as missing
(happy: 1 changes made)
\end{stlog}

\begin{stlog}
. recode age (100/max = 100) // age is topcoded at 100 (may or may not be plausible)
(age: 1 changes made)
\end{stlog}

\section{\texorpdfstring{We \texttt{describe} and \texttt{summarize} The
Data And See The Changes That Have Been
Made}{We describe and summarize The Data And See The Changes That Have Been Made}}\label{we-describe-and-summarize-the-data-and-see-the-changes-that-have-been-made}

\begin{stlog}
. describe
{\smallskip}
Contains data
  obs:           100                          
 vars:             3                          
\HLI{150}
              storage   display    value
variable name   type    format     label      variable label
\HLI{150}
id              float   \%9.0g                 ID
age             float   \%9.0g                 Age in Years
happy           float   \%9.0g      happy      Happiness Scale
\HLI{150}
Sorted by: 
     Note: Dataset has changed since last saved.
\end{stlog}

\begin{stlog}
. summarize
{\smallskip}
    Variable {\VBAR}        Obs        Mean    Std. Dev.       Min        Max
\HLI{13}{\PLUS}\HLI{57}
          id {\VBAR}        100        50.5    29.01149          1        100
         age {\VBAR}        100    48.90129    10.61974    24.7626        100
       happy {\VBAR}         99           3    1.442786          1          5
\end{stlog}

\end{document}
