% Options for packages loaded elsewhere
\PassOptionsToPackage{unicode}{hyperref}
\PassOptionsToPackage{hyphens}{url}
\documentclass[
]{article}
\usepackage{xcolor}
\usepackage{stata}
\usepackage[margin=1 in]{geometry}
\usepackage{amsmath,amssymb}
\setcounter{secnumdepth}{-\maxdimen} % remove section numbering
\usepackage{iftex}
\ifPDFTeX
  \usepackage[T1]{fontenc}
  \usepackage[utf8]{inputenc}
  \usepackage{textcomp} % provide euro and other symbols
\else % if luatex or xetex
  \usepackage{unicode-math} % this also loads fontspec
  \defaultfontfeatures{Scale=MatchLowercase}
  \defaultfontfeatures[\rmfamily]{Ligatures=TeX,Scale=1}
\fi
\usepackage{lmodern}
\ifPDFTeX\else
  % xetex/luatex font selection
\fi
% Use upquote if available, for straight quotes in verbatim environments
\IfFileExists{upquote.sty}{\usepackage{upquote}}{}
\IfFileExists{microtype.sty}{% use microtype if available
  \usepackage[]{microtype}
  \UseMicrotypeSet[protrusion]{basicmath} % disable protrusion for tt fonts
}{}
\makeatletter
\@ifundefined{KOMAClassName}{% if non-KOMA class
  \IfFileExists{parskip.sty}{%
    \usepackage{parskip}
  }{% else
    \setlength{\parindent}{0pt}
    \setlength{\parskip}{6pt plus 2pt minus 1pt}}
}{% if KOMA class
  \KOMAoptions{parskip=half}}
\makeatother
\setlength{\emergencystretch}{3em} % prevent overfull lines
\providecommand{\tightlist}{%
  \setlength{\itemsep}{0pt}\setlength{\parskip}{0pt}}
\usepackage{bookmark}
\IfFileExists{xurl.sty}{\usepackage{xurl}}{} % add URL line breaks if available
\urlstyle{same}
\hypersetup{
  pdftitle={Cleaning Data With Stata},
  pdfauthor={Andy Grogan-Kaylor},
  hidelinks,
  pdfcreator={LaTeX via pandoc}}

\title{Cleaning Data With Stata}
\author{Andy Grogan-Kaylor}
\date{2 Sep 2025 11:54:32}

\begin{document}
\maketitle

\section{Background}\label{background}

\begin{quote}
It sometimes seems like 80\% of our work as data analysts is cleaning
the data, while only 20\% is the actual analysis. Here are some Stata
commands that are useful in cleaning data.
\end{quote}

First we simulate some data to work with, and to clean.

\section{Simulate Some Data}\label{simulate-some-data}

\begin{quote}
This section is provided for illustration only, as it may be helpful to
see \emph{how} the data was simulated, and the decisions that went into
simulating the data. You may also \emph{safely ignore} this section if
you like.
\end{quote}

Show / Hide Data Simulation Code

\begin{stlog}
. clear all 
\end{stlog}

\begin{stlog}
. set obs 100 // 100 observations
Number of observations ({\bftt{_N}}) was 0, now 100.
\end{stlog}

\begin{stlog}
. generate id = _n // random id 
\end{stlog}

\begin{stlog}
. generate age = rnormal(50,10) // random generated age
\end{stlog}

\begin{stlog}
. replace age = 200 in 1 // someone is 200 years old!
(1 real change made)
\end{stlog}

\begin{stlog}
. generate happy = runiformint(1,5) // randomly generated happiness
\end{stlog}

\begin{stlog}
. replace happy = 999 in 10 // simulate a missing value
(1 real change made)
\end{stlog}

\begin{stlog}
. generate somethingelse = rnormal(0, 1) // something else!
\end{stlog}

\section{Look At Some Of The Data}\label{look-at-some-of-the-data}

\begin{stlog}
. list in 1/10 // list first 10 observations
{\smallskip}
     {\TLC}\HLI{35}{\TRC}
     {\VBAR} id        age   happy   somethi{\tytilde}e {\VBAR}
     {\LFTT}\HLI{35}{\RGTT}
  1. {\VBAR}  1        200       1      .06784 {\VBAR}
  2. {\VBAR}  2   60.42631       3    1.459888 {\VBAR}
  3. {\VBAR}  3   52.97712       4   -.7060056 {\VBAR}
  4. {\VBAR}  4   32.77868       5    .0529082 {\VBAR}
  5. {\VBAR}  5     42.708       1   -.2976007 {\VBAR}
     {\LFTT}\HLI{35}{\RGTT}
  6. {\VBAR}  6   58.61826       3   -.2165111 {\VBAR}
  7. {\VBAR}  7   47.60646       2    1.299809 {\VBAR}
  8. {\VBAR}  8   55.16549       5    .1418742 {\VBAR}
  9. {\VBAR}  9   31.87984       3    1.550615 {\VBAR}
 10. {\VBAR} 10   39.84875     999    .2515597 {\VBAR}
     {\BLC}\HLI{35}{\BRC}
\end{stlog}

\section{Clean The Data!}\label{clean-the-data}

\subsection{\texorpdfstring{Look at The Data and Think About The Data
(\texttt{describe}, \texttt{summarize}, \texttt{tabulate},
\texttt{codebook},
\texttt{browse})}{Look at The Data and Think About The Data (describe, summarize, tabulate, codebook, browse)}}\label{look-at-the-data-and-think-about-the-data-describe-summarize-tabulate-codebook-browse}

\begin{quote}
When we look at variables we are looking for values that don't make
sense, or that are outside the plausible range. As we are working with
the data, it may sometimes be helpful to \texttt{browse} the data.
\end{quote}

\begin{stlog}
. describe // describe the data
{\smallskip}
Contains data
 Observations:           100                  
    Variables:             4                  
\HLI{85}
Variable      Storage   Display    Value
    name         type    format    label      Variable label
\HLI{85}
id              float   \%9.0g                 
age             float   \%9.0g                 
happy           float   \%9.0g                 
somethingelse   float   \%9.0g                 
\HLI{85}
Sorted by: 
     Note: Dataset has changed since last saved.
\end{stlog}

\begin{stlog}
. summarize // descriptive statistics
{\smallskip}
    Variable {\VBAR}        Obs        Mean    Std. dev.       Min        Max
\HLI{13}{\PLUS}\HLI{57}
          id {\VBAR}        100        50.5    29.01149          1        100
         age {\VBAR}        100    49.85175    17.83505   24.33709        200
       happy {\VBAR}        100        12.9    99.61598          1        999
somethinge{\tytilde}e {\VBAR}        100   -.0085413    1.072256  -2.685881   2.794861
\end{stlog}

\begin{stlog}
. tabulate happy // tabulation of this particular categorical variable
{\smallskip}
      happy {\VBAR}      Freq.     Percent        Cum.
\HLI{12}{\PLUS}\HLI{35}
          1 {\VBAR}         21       21.00       21.00
          2 {\VBAR}         19       19.00       40.00
          3 {\VBAR}         23       23.00       63.00
          4 {\VBAR}         17       17.00       80.00
          5 {\VBAR}         19       19.00       99.00
        999 {\VBAR}          1        1.00      100.00
\HLI{12}{\PLUS}\HLI{35}
      Total {\VBAR}        100      100.00
\end{stlog}

\begin{stlog}
. codebook happy // VERY detailed view of this particular categorical variable
{\smallskip}
\HLI{85}
happy                                                                     (unlabeled)
\HLI{85}
{\smallskip}
                  Type: Numeric (float)
{\smallskip}
                 Range: [1,999]                       Units: 1
         Unique values: 6                         Missing .: 0/100
{\smallskip}
            Tabulation: Freq.  Value
                           21  1
                           19  2
                           23  3
                           17  4
                           19  5
                            1  999
\end{stlog}

Notice that\ldots{}

\begin{itemize}
\tightlist
\item
  There are variables in which we may not have interest.
\item
  None of the variables are labelled informatively.
\item
  Variables do not seem to have informative value labels.
\item
  Someone appears to 200 years old.
\item
  There appear to be missing values in the variable \texttt{happy} that
  need to be recoded.
\end{itemize}

\begin{quote}
Remember that the command \texttt{lookfor} is often very helpful in
\emph{looking for} a particular variable. e.g.~\texttt{lookfor\ happy}.
\end{quote}

\subsection{\texorpdfstring{Only \texttt{keep} The Variables Of
Interest}{Only keep The Variables Of Interest}}\label{only-keep-the-variables-of-interest}

We may only be interested in keeping some variables to keep our analytic
data set more manageable.

For this particular analysis we may wish to drop the variable called
\texttt{somethingelse}.

\begin{stlog}
. keep id age happy // keep only relevant variables
\end{stlog}

We could also have said \texttt{drop\ somethingelse}.

\subsection{\texorpdfstring{Add \emph{Variable} Labels
(\texttt{label\ variable\ "..."})}{Add Variable Labels (label variable "...")}}\label{add-variable-labels-label-variable-...}

\begin{stlog}
. label variable id "ID" // label variable
\end{stlog}

\begin{stlog}
. label variable age "Age in Years" // label variable
\end{stlog}

\begin{stlog}
. label variable happy "Happiness Scale" // label variable
\end{stlog}

\subsection{\texorpdfstring{Create \emph{Value} Labels
(\texttt{label\ define\ ...})}{Create Value Labels (label define ...)}}\label{create-value-labels-label-define-...}

\begin{stlog}
. label define happy 1 "Rarely" 2 "Sometimes" 3 "Often" 4 "Always" // create value la
> bel
\end{stlog}

\subsection{\texorpdfstring{Attach \emph{Value} Labels To
\emph{Variables}
(\texttt{label\ values\ ...})}{Attach Value Labels To Variables (label values ...)}}\label{attach-value-labels-to-variables-label-values-...}

\emph{Variables} and \emph{value labels} can have the same names but are
different things. We add the variable label \texttt{happy} to the
variable named \texttt{happy}.

\begin{stlog}
. label values happy happy // attach VALUE LABEL happy to VARIABLE happy
\end{stlog}

\subsection{\texorpdfstring{Recode Outliers, Values That Are Errors, Or
Values That Should Be Coded As Missing
(\texttt{recode})}{Recode Outliers, Values That Are Errors, Or Values That Should Be Coded As Missing (recode)}}\label{recode-outliers-values-that-are-errors-or-values-that-should-be-coded-as-missing-recode}

\begin{stlog}
. recode happy (999 = .) // recode values as missing
(1 changes made to {\bftt{happy}})
\end{stlog}

\begin{stlog}
. recode age (100/max = 100) // age is topcoded at 100 (may or may not be plausible)
(1 changes made to {\bftt{age}})
\end{stlog}

\section{\texorpdfstring{We \texttt{describe} and \texttt{summarize} The
Data And See The Changes That Have Been
Made}{We describe and summarize The Data And See The Changes That Have Been Made}}\label{we-describe-and-summarize-the-data-and-see-the-changes-that-have-been-made}

\begin{stlog}
. describe
{\smallskip}
Contains data
 Observations:           100                  
    Variables:             3                  
\HLI{85}
Variable      Storage   Display    Value
    name         type    format    label      Variable label
\HLI{85}
id              float   \%9.0g                 ID
age             float   \%9.0g                 Age in Years
happy           float   \%9.0g      happy      Happiness Scale
\HLI{85}
Sorted by: 
     Note: Dataset has changed since last saved.
\end{stlog}

\begin{stlog}
. summarize
{\smallskip}
    Variable {\VBAR}        Obs        Mean    Std. dev.       Min        Max
\HLI{13}{\PLUS}\HLI{57}
          id {\VBAR}        100        50.5    29.01149          1        100
         age {\VBAR}        100    48.85175    10.71257   24.33709        100
       happy {\VBAR}         99    2.939394    1.412901          1          5
\end{stlog}

\end{document}
