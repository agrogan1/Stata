\documentclass[11pt,nofonts,]{tufte-handout}

% ams
\usepackage{amssymb,amsmath}

\usepackage{ifxetex,ifluatex}
\usepackage{fixltx2e} % provides \textsubscript
\ifnum 0\ifxetex 1\fi\ifluatex 1\fi=0 % if pdftex
  \usepackage[T1]{fontenc}
  \usepackage[utf8]{inputenc}
\else % if luatex or xelatex
  \makeatletter
  \@ifpackageloaded{fontspec}{}{\usepackage{fontspec}}
  \makeatother
  \defaultfontfeatures{Ligatures=TeX,Scale=MatchLowercase}
  \makeatletter
  \@ifpackageloaded{soul}{
     \renewcommand\allcapsspacing[1]{{\addfontfeature{LetterSpace=15}#1}}
     \renewcommand\smallcapsspacing[1]{{\addfontfeature{LetterSpace=10}#1}}
   }{}
  \makeatother
\fi

% graphix
\usepackage{graphicx}
\setkeys{Gin}{width=\linewidth,totalheight=\textheight,keepaspectratio}

% booktabs
\usepackage{booktabs}

% url
\usepackage{url}

% hyperref
\usepackage{hyperref}

% units.
\usepackage{units}


\setcounter{secnumdepth}{2}

% citations
\usepackage{natbib}
\bibliographystyle{plainnat}

%% tint override
\setcitestyle{round} 

% pandoc syntax highlighting

% longtable

% multiplecol
\usepackage{multicol}

% strikeout
\usepackage[normalem]{ulem}

% morefloats
\usepackage{morefloats}


% tightlist macro required by pandoc >= 1.14
\providecommand{\tightlist}{%
  \setlength{\itemsep}{0pt}\setlength{\parskip}{0pt}}

% title / author / date
\title{Two Page Stata}
\author{Andrew Grogan-Kaylor}
\date{February 18, 2021}

%% -- tint overrides
%% fonts, using roboto (condensed) as default
\usepackage[sfdefault,condensed]{roboto}
%% also nice: \usepackage[default]{lato}

%% colored links, setting 'borrowed' from RJournal.sty with 'Thanks, Achim!'
\RequirePackage{color}
\definecolor{link}{rgb}{0.1,0.1,0.8} %% blue with some grey
\hypersetup{
  colorlinks,%
  citecolor=link,%
  filecolor=link,%
  linkcolor=link,%
  urlcolor=link
}

%% macros
\makeatletter

%% -- tint does not use italics or allcaps in title
\renewcommand{\maketitle}{%     
  \newpage
  \global\@topnum\z@% prevent floats from being placed at the top of the page
  \begingroup
    \setlength{\parindent}{0pt}%
    \setlength{\parskip}{4pt}%
    \let\@@title\@empty
    \let\@@author\@empty
    \let\@@date\@empty
    \ifthenelse{\boolean{@tufte@sfsidenotes}}{%
      %\gdef\@@title{\sffamily\LARGE\allcaps{\@title}\par}%
      %\gdef\@@author{\sffamily\Large\allcaps{\@author}\par}%
      %\gdef\@@date{\sffamily\Large\allcaps{\@date}\par}%
      \gdef\@@title{\begingroup\fontseries{b}\selectfont\LARGE{\@title}\par}%
      \gdef\@@author{\begingroup\fontseries{l}\selectfont\Large{\@author}\par}%
      \gdef\@@date{\begingroup\fontseries{l}\selectfont\Large{\@date}\par}%
    }{%
      %\gdef\@@title{\LARGE\itshape\@title\par}%
      %\gdef\@@author{\Large\itshape\@author\par}%
      %\gdef\@@date{\Large\itshape\@date\par}%
      \gdef\@@title{\begingroup\fontseries{b}\selectfont\LARGE\@title\par\endgroup}%
      \gdef\@@author{\begingroup\fontseries{l}\selectfont\Large\@author\par\endgroup}%
      \gdef\@@date{\begingroup\fontseries{l}\selectfont\Large\@date\par\endgroup}%
    }%
    \@@title
    \@@author
    \@@date
  \endgroup
  \thispagestyle{plain}% suppress the running head
  \tuftebreak% add some space before the text begins
  \@afterindentfalse\@afterheading% suppress indentation of the next paragraph
}

%% -- tint does not use italics or allcaps in section/subsection/paragraph
\titleformat{\section}%
  [hang]% shape
  %{\normalfont\Large\itshape}% format applied to label+text
  {\fontseries{b}\selectfont\Large}% format applied to label+text
  {\thesection}% label
  {1em}% horizontal separation between label and title body
  {}% before the title body
  []% after the title body

\titleformat{\subsection}%
  [hang]% shape
  %{\normalfont\large\itshape}% format applied to label+text
  {\fontseries{m}\selectfont\large}% format applied to label+text
  {\thesubsection}% label
  {1em}% horizontal separation between label and title body
  {}% before the title body
  []% after the title body

\titleformat{\paragraph}%
  [runin]% shape
  %{\normalfont\itshape}% format applied to label+text
  {\fontseries{l}\selectfont}% format applied to label+text
  {\theparagraph}% label
  {1em}% horizontal separation between label and title body
  {}% before the title body
  []% after the title body

%% -- tint does not use italics here either
% Formatting for main TOC (printed in front matter)
% {section} [left] {above} {before w/label} {before w/o label} {filler + page} [after]
\ifthenelse{\boolean{@tufte@toc}}{%
  \titlecontents{part}% FIXME
    [0em] % distance from left margin
    %{\vspace{1.5\baselineskip}\begin{fullwidth}\LARGE\rmfamily\itshape} % above (global formatting of entry)
    {\vspace{1.5\baselineskip}\begin{fullwidth}\fontseries{m}\selectfont\LARGE} % above (global formatting of entry)
    {\contentslabel{2em}} % before w/label (label = ``II'')
    {} % before w/o label
    {\rmfamily\upshape\qquad\thecontentspage} % filler + page (leaders and page num)
    [\end{fullwidth}] % after
  \titlecontents{chapter}%
    [0em] % distance from left margin
    %{\vspace{1.5\baselineskip}\begin{fullwidth}\LARGE\rmfamily\itshape} % above (global formatting of entry)
    {\vspace{1.5\baselineskip}\begin{fullwidth}\fontseries{m}\selectfont\LARGE} % above (global formatting of entry)
    {\hspace*{0em}\contentslabel{2em}} % before w/label (label = ``2'')
    {\hspace*{0em}} % before w/o label
    %{\rmfamily\upshape\qquad\thecontentspage} % filler + page (leaders and page num)
    {\upshape\qquad\thecontentspage} % filler + page (leaders and page num)
    [\end{fullwidth}] % after
  \titlecontents{section}% FIXME
    [0em] % distance from left margin
    %{\vspace{0\baselineskip}\begin{fullwidth}\Large\rmfamily\itshape} % above (global formatting of entry)
    {\vspace{0\baselineskip}\begin{fullwidth}\fontseries{m}\selectfont\Large} % above (global formatting of entry)
    {\hspace*{2em}\contentslabel{2em}} % before w/label (label = ``2.6'')
    {\hspace*{2em}} % before w/o label
    %{\rmfamily\upshape\qquad\thecontentspage} % filler + page (leaders and page num)
    {\upshape\qquad\thecontentspage} % filler + page (leaders and page num)
    [\end{fullwidth}] % after
  \titlecontents{subsection}% FIXME
    [0em] % distance from left margin
    %{\vspace{0\baselineskip}\begin{fullwidth}\large\rmfamily\itshape} % above (global formatting of entry)
    {\vspace{0\baselineskip}\begin{fullwidth}\fontseries{m}\selectfont\large} % above (global formatting of entry)
    {\hspace*{4em}\contentslabel{4em}} % before w/label (label = ``2.6.1'')
    {\hspace*{4em}} % before w/o label
    %{\rmfamily\upshape\qquad\thecontentspage} % filler + page (leaders and page num)
    {\upshape\qquad\thecontentspage} % filler + page (leaders and page num)
    [\end{fullwidth}] % after
  \titlecontents{paragraph}% FIXME
    [0em] % distance from left margin
    %{\vspace{0\baselineskip}\begin{fullwidth}\normalsize\rmfamily\itshape} % above (global formatting of entry)
    {\vspace{0\baselineskip}\begin{fullwidth}\fontseries{m}\selectfont\normalsize\rmfamily} % above (global formatting of entry)
    {\hspace*{6em}\contentslabel{2em}} % before w/label (label = ``2.6.0.0.1'')
    {\hspace*{6em}} % before w/o label
    %{\rmfamily\upshape\qquad\thecontentspage} % filler + page (leaders and page num)
    {\upshape\qquad\thecontentspage} % filler + page (leaders and page num)
    [\end{fullwidth}] % after
}{}

  
\makeatother



\begin{document}

\maketitle




An introduction to Stata\footnote{\href{https://twitter.com/intent/tweet?url=http\%3A\%2F\%2Fwww-personal.umich.edu\%2F\%7Eagrogan\%2Fstata\%2FTwoPageStata.pdf\&text=Two\%20Page\%20\%23Stata}{Share
  on Twitter}} in 2 pages.\footnote{Comments, questions and corrections
  most welcome and may be sent to:
  \href{http://www.umich.edu/~agrogan}{Andrew Grogan-Kaylor} @
  \url{agrogan@umich.edu}. This document available on the web @
  \url{https://agrogan1.github.io/Stata/}} Commands\footnote{The general
  idea of most Stata commands is
  \texttt{command\ variable(s),\ options}. Often it is not necessary to
  use any options since the authors of Stata have done such a good job
  of thinking about the defaults.} that you actually type\footnote{The
  Stata interface makes it extremely easy to do rapid interactive data
  analysis. Hit \textbf{PAGE-UP} to recall the most recent command,
  which you can then quickly edit and resubmit.} into Stata\footnote{Use
  the \textbf{DO FILE EDITOR} to save Stata commands that you want to
  use again, and to create an \emph{audit trail} of your work so that
  your workflow is \emph{documented} and \emph{replicable.}
  \texttt{log\ using\ filename,\ replace} will save a log file of your
  results. \texttt{log\ close} closes the log file.} are represented in
\texttt{monospace\ font}. x and y refer to variables in your data. The
treatment here is intended to be extremely brief, in order to create a
kind of ``cheat sheet'' that can be presented in 2 pages. More
documentation on any command is available in the printed or PDF Stata
manuals, or by typing \texttt{help\ command}.

\hypertarget{get-acquainted-with-your-datacodebook}{%
\section[Get Acquainted With Your Data]{\texorpdfstring{Get Acquainted
With Your Data\footnote{\texttt{codebook\ x\ y} will produce a nicely
  formatted codebook of selected variables, which is especially useful
  if you have added variable labels with the label variable command.
  \texttt{codebook} is especially useful for seeing how numerical values
  are associated with value labels. \texttt{codebook} by itself will
  list every variable in your data and generate a lot of {[}probably too
  much{]} output.}}{Get Acquainted With Your Data}}\label{get-acquainted-with-your-datacodebook}}

\texttt{lookfor} allows you to find variables that contain a specified
keyword. This is especially useful in large data sets with many
variables. Often abbreviated keywords are the most helpful. e.g.~to find
a poverty variable, type \texttt{lookfor\ pov}.

With very large data sets, it may be helpful to use
\texttt{keep\ x\ y\ z} to only keep the variables with which you are
working.

\texttt{describe} tells you about the contents of a specific variable.
E.g. \texttt{describe\ x\ y}. \texttt{describe,\ short} will tell you
very basic things about your data, including the number of observations
in the data set, and the size of your data file.

\hypertarget{process-your-datamissing-values}{%
\section[Process Your Data]{\texorpdfstring{Process Your Data\footnote{Data
  with missing values, often represented as negative numbers (e.g.~-99,
  -9, -8) need to be recoded so that the missing values are represented
  as a missing value character (``.'') that Stata knows to exclude from
  calculations.}}{Process Your Data}}\label{process-your-datamissing-values}}

\texttt{recode\ x\ (oldvalue\ =\ newvalue),\ generate(z)} will recode a
variable into a new variable, often a good idea.

\texttt{recode\ \_all\ (-99/-1\ =\ .)} will recode all negative numbers
from -99 to -1 to missing for all variables in your data.
\texttt{recode\ x\ (7/9\ =\ .)} changes 7 through 9 to be missing for x.
Indeed, \texttt{recode} will change specific values in your data to
anything you want, not just missing values.

It is often convenient to \texttt{rename} your variables so that the
variables have more intuitively understandable names
e.g.~\texttt{rename\ x\ depression}.

You can create new variables out of old variables using
\texttt{generate\ newvar\ =\ expression}
e.g.~\texttt{generate\ newvar\ =\ oldvar1\ +\ oldvar2}.\footnote{\texttt{alpha\ oldvar1\ oldvar2}
  will calculate Cronbach's alpha from this scale.}

It is sometimes useful to \texttt{sort} your data. \texttt{sort\ x} will
sort your data by the values of x.

\hypertarget{descriptive-statistics}{%
\section{Descriptive Statistics}\label{descriptive-statistics}}

\texttt{summarize} gives you basic descriptive statistics for a
variable, such as the mean (average). Especially useful for continuous
variables. E.g. \texttt{summarize\ x\ y} or
\texttt{summarize\ x\ y,\ detail}.

\texttt{tabulate} gives you a frequency distribution for your variable.
Especially useful for categorical variables. e.g.~\texttt{tabulate\ x}.

\hypertarget{bivariate-statisticsanova}{%
\section[Bivariate Statistics]{\texorpdfstring{Bivariate
Statistics\footnote{\texttt{oneway\ continous\_var\ categorical\_var,\ tabulate}
  gives you a oneway ANOVA of a continuous variable over a categorical
  variable.}}{Bivariate Statistics}}\label{bivariate-statisticsanova}}

Tabulating two categorical variables together gives you a
cross-tabulation of those variables, e.g
\texttt{tabulate\ x\ y,\ row\ col\ chi2}

\texttt{pwcorr\ x\ y,\ sig} gives you the pairwise correlation of two
continuous variables.

\hypertarget{multivariate-statistics}{%
\section{Multivariate Statistics}\label{multivariate-statistics}}

\texttt{regress\ y\ x} regresses y on x.\footnote{After running many
  multivariate models \texttt{estat\ summarize} will give you simple
  descriptive statistics for the specific sample used in that particular
  analysis.}\\
\texttt{regress\ y\ x\ z} regresses y on x and z.\footnote{Other
  regression commands follow a very similar format:
  \texttt{command\ y\ x\ z} but are beyond the purview of this 2 page
  guide.} \texttt{regress\ y\ x\ i.z} regresses y on x and z, treating x
as continuous and z as a set of categorical indicator
variables.\footnote{\texttt{i.x} is Stata's notation for treating
  independent variables as \emph{categorical} or \emph{indicator}
  variables.} \texttt{regress\ y\ c.x\#\#i.z} regresses y on continuous
x and categorical z, providing both main effects for x and z and the
interaction of x and z.

\hypertarget{graphinggraph-options}{%
\section[Graphing]{\texorpdfstring{Graphing\footnote{For all graphs,
  options after a ``,'' will be helpful in titling your graph
  e.g.~\texttt{twoway\ lfit\ y\ x,\ title("...")\ xtitle("...")\ ytitle("...")}}}{Graphing}}\label{graphinggraph-options}}

\texttt{histogram\ x} will give you a nice display of one variable.
\texttt{histogram\ x,\ by(y)} may be useful for comparing the
distributions of two variables over the categories of y.

\texttt{histogram\ x,\ percent} will scale the y-axis more intuitively
in terms of percentages. \texttt{histogram\ x,\ discrete}\footnote{The
  percent and discrete options can be combined.} gives a nicer display
for categorical variables.

\texttt{twoway\ scatter\ y\ x} gives you a scatterplot of your data.
\texttt{twoway\ lfit\ y\ x} will give you a linear fit graph. The two
syntaxes may be combined
e.g.~\texttt{twoway\ (scatter\ y\ x)\ (lfit\ y\ x)}.

\texttt{graph\ bar,\ over(x)} is useful for creating a bar graph of the
counts of a categorical variable x. \texttt{graph\ bar\ y,\ over(x)}
will create a bar graph of the means of y over categories of x.

\hypertarget{by-and-bysort}{%
\section{\texorpdfstring{\texttt{by:} and
\texttt{bysort:}}{by: and bysort:}}\label{by-and-bysort}}

In many cases you may want to look at the results of some calculation
for x, or x and y over a third variable z. In such cases the by: syntax
will be especially useful. For example to look at the correlation of x
and y over different values of z.

\texttt{bysort\ z:\ pwcorr\ x\ y,\ sig}



\end{document}
