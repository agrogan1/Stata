% Options for packages loaded elsewhere
\PassOptionsToPackage{unicode}{hyperref}
\PassOptionsToPackage{hyphens}{url}
\PassOptionsToPackage{dvipsnames,svgnames,x11names}{xcolor}
%
\documentclass[
]{report}

\usepackage{amsmath,amssymb}
\usepackage{lmodern}
\usepackage{iftex}
\ifPDFTeX
  \usepackage[T1]{fontenc}
  \usepackage[utf8]{inputenc}
  \usepackage{textcomp} % provide euro and other symbols
\else % if luatex or xetex
  \usepackage{unicode-math}
  \defaultfontfeatures{Scale=MatchLowercase}
  \defaultfontfeatures[\rmfamily]{Ligatures=TeX,Scale=1}
\fi
% Use upquote if available, for straight quotes in verbatim environments
\IfFileExists{upquote.sty}{\usepackage{upquote}}{}
\IfFileExists{microtype.sty}{% use microtype if available
  \usepackage[]{microtype}
  \UseMicrotypeSet[protrusion]{basicmath} % disable protrusion for tt fonts
}{}
\makeatletter
\@ifundefined{KOMAClassName}{% if non-KOMA class
  \IfFileExists{parskip.sty}{%
    \usepackage{parskip}
  }{% else
    \setlength{\parindent}{0pt}
    \setlength{\parskip}{6pt plus 2pt minus 1pt}}
}{% if KOMA class
  \KOMAoptions{parskip=half}}
\makeatother
\usepackage{xcolor}
\usepackage[top=1in,left=1in,right=1in]{geometry}
\setlength{\emergencystretch}{3em} % prevent overfull lines
\setcounter{secnumdepth}{-\maxdimen} % remove section numbering
% Make \paragraph and \subparagraph free-standing
\ifx\paragraph\undefined\else
  \let\oldparagraph\paragraph
  \renewcommand{\paragraph}[1]{\oldparagraph{#1}\mbox{}}
\fi
\ifx\subparagraph\undefined\else
  \let\oldsubparagraph\subparagraph
  \renewcommand{\subparagraph}[1]{\oldsubparagraph{#1}\mbox{}}
\fi


\providecommand{\tightlist}{%
  \setlength{\itemsep}{0pt}\setlength{\parskip}{0pt}}\usepackage{longtable,booktabs,array}
\usepackage{calc} % for calculating minipage widths
% Correct order of tables after \paragraph or \subparagraph
\usepackage{etoolbox}
\makeatletter
\patchcmd\longtable{\par}{\if@noskipsec\mbox{}\fi\par}{}{}
\makeatother
% Allow footnotes in longtable head/foot
\IfFileExists{footnotehyper.sty}{\usepackage{footnotehyper}}{\usepackage{footnote}}
\makesavenoteenv{longtable}
\usepackage{graphicx}
\makeatletter
\def\maxwidth{\ifdim\Gin@nat@width>\linewidth\linewidth\else\Gin@nat@width\fi}
\def\maxheight{\ifdim\Gin@nat@height>\textheight\textheight\else\Gin@nat@height\fi}
\makeatother
% Scale images if necessary, so that they will not overflow the page
% margins by default, and it is still possible to overwrite the defaults
% using explicit options in \includegraphics[width, height, ...]{}
\setkeys{Gin}{width=\maxwidth,height=\maxheight,keepaspectratio}
% Set default figure placement to htbp
\makeatletter
\def\fps@figure{htbp}
\makeatother

\makeatletter
\makeatother
\makeatletter
\makeatother
\makeatletter
\@ifpackageloaded{caption}{}{\usepackage{caption}}
\AtBeginDocument{%
\ifdefined\contentsname
  \renewcommand*\contentsname{Table of contents}
\else
  \newcommand\contentsname{Table of contents}
\fi
\ifdefined\listfigurename
  \renewcommand*\listfigurename{List of Figures}
\else
  \newcommand\listfigurename{List of Figures}
\fi
\ifdefined\listtablename
  \renewcommand*\listtablename{List of Tables}
\else
  \newcommand\listtablename{List of Tables}
\fi
\ifdefined\figurename
  \renewcommand*\figurename{Figure}
\else
  \newcommand\figurename{Figure}
\fi
\ifdefined\tablename
  \renewcommand*\tablename{Table}
\else
  \newcommand\tablename{Table}
\fi
}
\@ifpackageloaded{float}{}{\usepackage{float}}
\floatstyle{ruled}
\@ifundefined{c@chapter}{\newfloat{codelisting}{h}{lop}}{\newfloat{codelisting}{h}{lop}[chapter]}
\floatname{codelisting}{Listing}
\newcommand*\listoflistings{\listof{codelisting}{List of Listings}}
\makeatother
\makeatletter
\@ifpackageloaded{caption}{}{\usepackage{caption}}
\@ifpackageloaded{subcaption}{}{\usepackage{subcaption}}
\makeatother
\makeatletter
\@ifpackageloaded{tcolorbox}{}{\usepackage[many]{tcolorbox}}
\makeatother
\makeatletter
\@ifundefined{shadecolor}{\definecolor{shadecolor}{rgb}{.97, .97, .97}}
\makeatother
\makeatletter
\makeatother
\ifLuaTeX
  \usepackage{selnolig}  % disable illegal ligatures
\fi
\IfFileExists{bookmark.sty}{\usepackage{bookmark}}{\usepackage{hyperref}}
\IfFileExists{xurl.sty}{\usepackage{xurl}}{} % add URL line breaks if available
\urlstyle{same} % disable monospaced font for URLs
\hypersetup{
  pdftitle={Two Page Stata},
  pdfauthor={Andrew Grogan-Kaylor},
  colorlinks=true,
  linkcolor={blue},
  filecolor={Maroon},
  citecolor={Blue},
  urlcolor={Blue},
  pdfcreator={LaTeX via pandoc}}

\title{Two Page Stata}
\usepackage{etoolbox}
\makeatletter
\providecommand{\subtitle}[1]{% add subtitle to \maketitle
  \apptocmd{\@title}{\par {\large #1 \par}}{}{}
}
\makeatother
\subtitle{An Introduction to Stata}
\author{Andrew Grogan-Kaylor}
\date{1/30/23}

\begin{document}
\maketitle
\ifdefined\Shaded\renewenvironment{Shaded}{\begin{tcolorbox}[interior hidden, frame hidden, enhanced, boxrule=0pt, breakable, borderline west={3pt}{0pt}{shadecolor}, sharp corners]}{\end{tcolorbox}}\fi

\hypertarget{introduction}{%
\section{Introduction}\label{introduction}}

An introduction to Stata in 2 pages.\footnote{Comments, questions and
  corrections most welcome and may be sent to:
  \href{https://agrogan1.github.io/}{Andrew Grogan-Kaylor} @
  \url{agrogan@umich.edu}. This document available on the web @
  \url{https://agrogan1.github.io/Stata/}} Commands that you actually
type into Stata are represented in \texttt{monospace\ font}. x and y
refer to variables in your data. The treatment here is intended to be
extremely brief, in order to create a kind of ``cheat sheet'' that can
be presented in 2 pages. More documentation on any command is available
in the printed or PDF Stata manuals, or by typing
\texttt{help\ command}.

The general idea of most Stata commands is
\texttt{command\ variable(s),\ options}. Often it is not necessary to
use any options since the authors of Stata have done such a good job of
thinking about the defaults.

The Stata interface makes it extremely easy to do rapid interactive data
analysis. Hit \textbf{PAGE-UP} to recall the most recent command, which
you can then quickly edit and resubmit.

Use the \textbf{DO FILE EDITOR} to save Stata commands that you want to
use again, and to create an \emph{audit trail} of your work so that your
workflow is \emph{documented} and \emph{replicable.}
\texttt{log\ using\ filename,\ replace} will save a log file of your
results. \texttt{log\ close} closes the log file.

\hypertarget{get-acquainted-with-your-data}{%
\section{Get Acquainted With Your
Data}\label{get-acquainted-with-your-data}}

\texttt{codebook\ x\ y} will produce a nicely formatted codebook of
selected variables, which is especially useful if you have added
variable labels and value labels. \texttt{codebook} is especially useful
for seeing how numerical values are associated with value labels.
\texttt{codebook} by itself will list every variable in your data and
generate a lot of {[}probably too much{]} output.

\texttt{lookfor} allows you to find variables that contain a specified
keyword. This is especially useful in large data sets with many
variables. Often abbreviated keywords are the most helpful. e.g.~to find
a poverty variable, type \texttt{lookfor\ pov}.

With very large data sets, it may be helpful to use
\texttt{keep\ x\ y\ z} to only keep the variables with which you are
working.

\texttt{describe} tells you about the contents of a specific variable.
E.g. \texttt{describe\ x\ y}. \texttt{describe,\ short} will tell you
very basic things about your data, including the number of observations
in the data set, and the size of your data file.

\hypertarget{process-your-data}{%
\section{Process Your Data}\label{process-your-data}}

Data with missing values, often represented as negative numbers
(e.g.~-99, -9, -8) need to be recoded so that the missing values are
represented as a missing value character (``.'') that Stata knows to
exclude from calculations.

\texttt{recode\ x\ (oldvalue\ =\ newvalue),\ generate(z)} will recode a
variable into a new variable, often a good idea.

\texttt{recode\ \_all\ (-99/-1\ =\ .)} will recode all negative numbers
from -99 to -1 to missing for all variables in your data.
\texttt{recode\ x\ (7/9\ =\ .)} changes 7 through 9 to be missing for x.
Indeed, \texttt{recode} will change specific values in your data to
anything you want, not just missing values.\footnote{\texttt{encode\ x,\ generate(x\_NUMERIC)}
  is often useful to create a \emph{numeric} version of \emph{string}
  variables.}

It is often convenient to \texttt{rename} your variables so that the
variables have more intuitively understandable names
e.g.~\texttt{rename\ x\ depression}.

You can create new variables out of old variables using
\texttt{generate\ newvar\ =\ expression}
e.g.~\texttt{generate\ newvar\ =\ oldvar1\ +\ oldvar2}.\footnote{\texttt{alpha\ oldvar1\ oldvar2}
  will calculate Cronbach's alpha from this scale.}

It is sometimes useful to \texttt{sort} your data. \texttt{sort\ x} will
sort your data by the values of x.

\hypertarget{descriptive-statistics}{%
\section{Descriptive Statistics}\label{descriptive-statistics}}

\texttt{summarize} gives you basic descriptive statistics for a
variable, such as the mean (average). Especially useful for continuous
variables. E.g. \texttt{summarize\ x\ y} or
\texttt{summarize\ x\ y,\ detail}.

\texttt{tabulate} gives you a frequency distribution for your variable.
Especially useful for categorical variables. e.g.~\texttt{tabulate\ x}.

\hypertarget{bivariate-statistics}{%
\section{Bivariate Statistics}\label{bivariate-statistics}}

Tabulating two categorical variables together gives you a
cross-tabulation of those variables, e.g
\texttt{tabulate\ x\ y,\ row\ col\ chi2}

\texttt{pwcorr\ x\ y,\ sig} gives you the pairwise correlation of two
continuous variables.

\texttt{oneway\ x\ z,\ tabulate} gives you a oneway ANOVA of continuous
variable x over categorical variable z.

\hypertarget{multivariate-statistics}{%
\section{Multivariate Statistics}\label{multivariate-statistics}}

\texttt{regress\ y\ x} regresses y on x.\footnote{After running many
  multivariate models \texttt{estat\ summarize} will give you simple
  descriptive statistics for the specific sample used in that particular
  analysis.} \texttt{regress\ y\ x\ z} regresses y on x and
z.\footnote{Other regression commands follow a very similar format:
  \texttt{command\ y\ x\ z} but are beyond the purview of this 2 page
  guide.} \texttt{regress\ y\ x\ i.z} regresses y on x and z, treating x
as continuous and z as a set of categorical indicator
variables.\footnote{\texttt{i.x} is Stata's notation for treating
  independent variables as \emph{categorical} or \emph{indicator}
  variables.} \texttt{regress\ y\ c.x\#\#i.z} regresses y on continuous
x and categorical z, providing both main effects for x and z and the
interaction of x and z.

\hypertarget{graphinggraph-options}{%
\section[Graphing]{\texorpdfstring{Graphing\footnote{For all graphs,
  options after a ``,'' will be helpful in titling your graph
  e.g.~\texttt{twoway\ lfit\ y\ x,\ title("...")\ xtitle("...")\ ytitle("...")}}}{Graphing}}\label{graphinggraph-options}}

\texttt{histogram\ x} will give you a nice display of one
variable.\footnote{\texttt{histogram\ x,\ percent} will scale the y-axis
  more intuitively in terms of percentages.
  \texttt{histogram\ x,\ discrete} gives a nicer display for categorical
  variables. The \texttt{percent} and \texttt{discrete} options can be
  combined.}

\texttt{twoway\ scatter\ y\ x} gives you a scatterplot of your data.
\texttt{twoway\ lfit\ y\ x} will give you a linear fit graph. The two
syntaxes may be combined
e.g.~\texttt{twoway\ (scatter\ y\ x)\ (lfit\ y\ x)}.

\texttt{graph\ bar,\ over(x)} is useful for creating a bar graph of the
counts of a categorical variable x. \texttt{graph\ bar\ y,\ over(x)}
will create a bar graph of the means of y over categories of x.

\hypertarget{by-and-bysort}{%
\section{\texorpdfstring{\texttt{by:} and
\texttt{bysort:}}{by: and bysort:}}\label{by-and-bysort}}

In many cases you may want to look at the results of some calculation
for x, or x and y over a third variable z. In such cases the by: syntax
will be especially useful. For example to look at the correlation of x
and y over different values of z:
\texttt{bysort\ z:\ pwcorr\ x\ y,\ sig}



\end{document}
